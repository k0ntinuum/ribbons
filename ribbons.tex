

\documentclass{article}
\usepackage[utf8]{inputenc}
\usepackage{setspace}
\usepackage{ mathrsfs }
\usepackage{amssymb} %maths
\usepackage{amsmath} %maths
\usepackage[margin=0.2in]{geometry}
\usepackage{graphicx}
\usepackage{ulem}
\setlength{\parindent}{0pt}
\setlength{\parskip}{10pt}
\usepackage{hyperref}
\usepackage[autostyle]{csquotes}

\usepackage{cancel}
\renewcommand{\i}{\textit}
\renewcommand{\b}{\textbf}
\newcommand{\q}{\enquote}
\newcommand{\p}{$\phi \ $}

\renewcommand{\H}{\Bbb H}
%\renewcommand{\l}[1]{\lceil #1 \rceil }
\renewcommand{\l}[1]{( #1 ) }
%\newcommand{\lte}{\sqsubset}
%\newcommand{\lte}{\angle}
\newcommand{\lte}{\preceq}
\newcommand{\ribbons}{\Bbb I}
\newcommand{\forks}{ \sqsubset}
\newcommand{\ident}{I}
\renewcommand{\center}{\mathring f  }

\newcommand{\N}{ \Bbb N}

%\vskip1.0in



\begin{document}
{\setstretch{0.0}{

\begin{huge}
RIBBONS

\section{Definition}

Let $\Bbb M$ (for mirror) be the set of all nonzero integers.

A function $f : \Bbb M \to \Bbb Q$ is a \b{ribbon} when 

$0 <  f(1) < f(2) < f(3) < ... < f(-3) < f(-2) < f(-1)$

and

$f(-n) - f(n) \longrightarrow 0$ as $n \longrightarrow \infty$. 

Denote the set of all such ribbons by $\ribbons$.

\section{An Example}

For $n \in \N$, define $I(n) = \frac{n}{n+1}$ and $I(-n) = [I(n)]^{-1}$. 

Then  $I = ...\frac{6}{5},\frac{5}{4},\frac{4}{3},\frac{3}{2},\frac{2}{1},,\frac{1}{2},\frac{2}{3},\frac{3}{4},\frac{4}{5},\frac{5}{6},... $ is a ribbon.\\

Note the repeated commas that mark the absent center, since $I$ is not defined at zero. 

Note also that $I$ is (will turn out to be ) a multiplicative identity.


\section{Order }

Define $f \forks g$ if $n > 0 \implies g(n) \le f(n) < f(-n) \le g(-n)$. 

Then define $f \sim  g$ if $\exists h \in \ribbons$ such that $ h \forks f$ and $h \sqsubset g $. 

In other words, $f$ and $g$ are equivalent if there is some $h$ that \q{forks} or \q{pierces} them both.


Also define $f < g$ if there $\exists n > 0$ such that $ f(-n) < g(n)$. 

This gives us $f < g, f>g$, or $f \sim g$ for any $f,g \in \ribbons$ 

\section{Addition and Multiplication}

Define $f + g$ by $(f + g)(z) = f(z) + g(z)$.

Define $fg$ by $(fg)(z) = f(z)g(z)$. 

Then $f \sim f', g \sim g' \implies f + g\sim f' + g'$.

Also $f \sim f', g \sim g' \implies fg \sim f'g'$.\\


\section{A Multiplicative Inverse}

For $f \in \ribbons$, define $f^*$ by $f^*(z) =  [ f(-z) ]^{-1} $. Then $f^* \in \ribbons$ and $ff^* \sim \ident$.

\section{Injecting Positive Rationals}

We can inject any positive $q \in \Bbb Q$ into the ribbons by scaling I. 

Define $[q] \in \ribbons$ by $ [q](n) = \frac{n}{n+1}q$ and $[q](-n) = [q(n)]^{-1}$.



\end{huge}

}}



\end{document}
